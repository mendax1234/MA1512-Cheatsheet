\documentclass[10pt, landscape]{article}
\usepackage[scaled=0.92]{helvet}
\usepackage{calc}
\usepackage{multicol}
\usepackage{ifthen}
\usepackage[a4paper,margin=3mm,landscape]{geometry}
\usepackage{amsmath,amsthm,amsfonts,amssymb}
\usepackage{color,graphicx,overpic}
\usepackage{hyperref}
\usepackage{newtxtext} 
\usepackage{enumitem}
\usepackage{amssymb}
\usepackage[table]{xcolor}
\usepackage{vwcol}
\usepackage{tikz}
\usetikzlibrary{arrows.meta}
\usetikzlibrary{calc}
\usepackage{mathtools}
\usepackage{nicematrix}
%For pictures / figures
\usepackage{color,graphicx,overpic}
\graphicspath{ {./images/} }
% for relations
\usepackage{cancel}
\usepackage{ mathrsfs }
\graphicspath{ {./images/} }
\setlist{nosep}

\pdfinfo{
  /Title (MA1512.pdf)
  /Creator (TeX)
  /Producer (pdfTeX 1.40.0)
  /Author (Seamus)
  /Subject (Example)
  /Keywords (pdflatex, latex,pdftex,tex)}

% Turn off header and footer
\pagestyle{empty}

\newenvironment{tightcenter}{%
  \setlength\topsep{0pt}
  \setlength\parskip{0pt}
  \begin{center}
}{%
  \end{center}
}

% redefine section commands to use less space
\makeatletter
\renewcommand{\section}{\@startsection{section}{1}{0mm}%
                                {-1ex plus -.5ex minus -.2ex}%
                                {0.5ex plus .2ex}%x
                                {\normalfont\large\bfseries}}
\renewcommand{\subsection}{\@startsection{subsection}{2}{0mm}%
                                {-1explus -.5ex minus -.2ex}%
                                {0.5ex plus .2ex}%
                                {\normalfont\normalsize\bfseries}}
\renewcommand{\subsubsection}{\@startsection{subsubsection}{3}{0mm}%
                                {-1ex plus -.5ex minus -.2ex}%
                                {1ex plus .2ex}%
                                {\normalfont\small\bfseries}}%
\renewcommand{\familydefault}{\sfdefault}
\renewcommand\rmdefault{\sfdefault}
% makes nested numbering (e.g. 1.1.1, 1.1.2, etc)
\renewcommand{\labelenumii}{\theenumii}
\renewcommand{\theenumii}{\theenumi.\arabic{enumii}.}
\renewcommand\labelitemii{•}
%  for logical not operator
\renewcommand{\lnot}{\mathord{\sim}}
\renewcommand{\bf}[1]{\textbf{#1}}
\newcommand{\abs}[1]{\vert #1 \vert}
\newcommand{\Mod}[1]{\ \mathrm{mod}\ #1}

\makeatother
\definecolor{myblue}{cmyk}{1,.72,0,.38}
\everymath\expandafter{\the\everymath \color{myblue}}
% Define BibTeX command
\def\BibTeX{{\rm B\kern-.05em{\sc i\kern-.025em b}\kern-.08em
    T\kern-.1667em\lower.7ex\hbox{E}\kern-.125emX}}
\let\iff\leftrightarrow
\let\Iff\Leftrightarrow
\let\then\rightarrow
\let\Then\Rightarrow

% Don't print section numbers
\setcounter{secnumdepth}{0}

\setlength{\parindent}{0pt}
\setlength{\parskip}{0pt plus 0.5ex}
%% this changes all items (enumerate and itemize)
\setlength{\leftmargini}{0.5cm}
\setlength{\leftmarginii}{0.5cm}
\setlist[itemize,1]{leftmargin=2mm,labelindent=1mm,labelsep=1mm}
\setlist[itemize,2]{leftmargin=4mm,labelindent=1mm,labelsep=1mm}

%My Environments
\newtheorem{example}[section]{Example}
% -----------------------------------------------------------------------

\begin{document}
\raggedright
\footnotesize
\begin{multicols}{4}


% multicol parameters
% These lengths are set only within the two main columns
\setlength{\columnseprule}{0.25pt}
\setlength{\premulticols}{1pt}
\setlength{\postmulticols}{1pt}
\setlength{\multicolsep}{1pt}
\setlength{\columnsep}{2pt}

\begin{center}
    \fbox{%
        \parbox{0.8\linewidth}{\centering \textcolor{black}{
            {\Large\textbf{MA1512}}
            \\ \normalsize{AY24/25 sem 1}}
            \\ {\footnotesize \textcolor{myblue}{github.com/mendax1234}} 
        }%
    }
\end{center}

\section{01. Introduction to Differential Equations}
\subsection{First Principles}
\begin{enumerate}
    \item (\textbf{Differential Equation}) Let $x$ be an independent variable and $y$ be a dependent variable. An equation that involves $x,y$ and \textbf{various derivatives of $y$} is called a \textbf{differential equation}. e.g. $(\frac{dy}{dx})^3+e^x+2=\frac{d^2y}{dx^2}$
    \item (\textbf{Ordinary Differential Equation}) In general, an equation of the form $F(x,y,\frac{dy}{dx}, \cdots, \frac{d^ny}{dx^n})=0$ is an \textbf{ordinary differential equation}. It is called so because there is only \textbf{one} independent variable and only \textbf{ordinary derivatives (not partial derivatives)} are involved.
    \item (\textbf{Order of a Differential Equation}) The \textbf{order} of a differential equation is the order of the \textbf{highest derivative} appearing in the differential equation. e.g. $dy/dx$ is first order derivative, $d^2y/dx^2$ is second order derivative.
    \item (\textbf{General Solution}) A \textbf{general solution} to a differential equation is a family of infinitely many possible solutions, often involving \textbf{arbitrary constants} and they satisfy the differential equation when they are substituted into the differential equation.
    \item (\textbf{Particular Solution}) With additional information such as \textit{initial condition} (where a differential equation is required to satisfy conditions on the dependent variable and its derivatives specified at one value of the independent variable), we can determine a \textbf{particular solution} that no longer involves arbitrary constants.
    \item (\textbf{The method of seperation of variables}) A first-order differential equation of the form $\frac{dy}{dx}=F(x,y)$ is \textbf{seperable} if it can be written as $M(x)dx=N(y)dy$. To solve this, directly integrate both sides of the equation, we will get $\int N(y)dy=\int M(x)dx+C$, where $C$ is an arbitrary constant.
    \begin{itemize}
        \item (\textbf{The position of arbitrary constant $C$}) The arbitrary constant must be added immediately when you integrate the independent variable.
        \item (\textbf{Notation}) Sometimes $dy/dx$ is written simply as $y'$.
        \item (\textbf{Some useful substitution})
        \begin{itemize}
            \item If $y'=f(ax+by+c)$, we employ a \textit{linear change of variable}. Let $u=ax+by+c\rightarrow u'=a+by'$
            \item If $y'=f(y/x)$, we let $y=xv$, and $y'=xv'+v$
            Note that in both substitutions, we assume the function at the right side can be written as $f(\cdots)$, which is the soul in substitution.
        \end{itemize}
    \end{itemize}
\end{enumerate}

\end{multicols}

% Dividing Line
\hrulefill \\

\begin{multicols}{3}
\end{multicols}

\end{document}